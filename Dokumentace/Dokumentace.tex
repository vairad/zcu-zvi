\documentclass[12pt, a4paper]{report}
\usepackage[utf8]{inputenc}
\usepackage[IL2]{fontenc}
\usepackage[czech]{babel}
\usepackage{graphicx}
\usepackage{epstopdf}
\usepackage{url}

\begin{document}
	\begin{titlepage}
	\includegraphics[width=5cm,natwidth=601,natheight=314]{logo.png}
		
	\vspace{4cm}
		\begin {center}
		{\Huge SEMESTRÁLNÍ PRÁCE\\ Z PŘEDMĚTU KIV/ZVI\\}
		{\huge Analýza sekvence mikroskopických snímků, segmentace, detekce objektů\\}
		\end {center}
	\vspace{7cm}
			
	\noindent vypracovali: Denisa Tarantíková, Radek Vais \\
				studijní čísla: A13B0445P, A13B0457P\\
				email:	denitara@students.zcu.cz, vaisr@students.zcu.cz\\
				datum:	14. 6. 2016
	\end{titlepage}

\tableofcontents
\chapter{Úvod}
	Cílem této semestrální práce bude analyzovat sekvenci mikroskopických snímků a detekovat v~nich jevy zvané konvoluce. Materiál, velikost řezu, zvětšení apod. Popis konvoluce, je i v jiných materiálech (s jiným tvarem). Výstupem práce bude kromě aplikace pro detekci konvolucí napsané v~jazyce C++ i soubor ve formátu *.xml, ve kterém budou uvedeny souřadnice konvolucí a snímky, na kterých se vyskytují.
\chapter{Zadání}
Obecné pokyny a pravidla pro vypracování semestrální práce:
\begin{enumerate}
	\item{Semestrální práce bude vypracována na téma podle zadání.}
	\item{Součástí semestrální práce je teoretické řešení zadaného úkolu a programová realizace, která
může být napsána v libovolném vhodném programovacím jazyku (Matlab nedoporučujeme).}
	\item Programová realizace by měla obsahovat základní funkce:
	\begin{itemize}
		\item{výběr souboru vstupního snímku [formát *.BMP, *.JPG,…] a jeho zobrazení}
		\item{zobrazování dílčích snímků jako výstupů jednotlivých fází zpracování úlohy včetně výsledku
konečného}
		\item{volba pořadí jednotlivých metod předzpracování (pokud bude typ zadané úlohy vyžadovat
předzpracování) a vlastního zpracování úlohy dle zadání}
		\item{navržené algoritmy budou optimalizovány podle časového kritéria}
		\item{funkce "Krok zpět", minimálně o 1 krok (podle charakteru úlohy)}
		\item{uložení výsledného snímku do výstupního obrazového souboru, formát viz vstupní snímek}
		\item{uložení výsledných hodnot do výstupních souborů v předepsaném a komentovaném formátu,
např. tabulky hodnot, příznakové vektory,…}
	\end{itemize}

	\item Pokud to zadané téma vyžaduje, bude součástí programové realizace také soubor metod pro
zobrazení globálních charakteristik a předzpracování snímků, tj. metody typu:
	\begin{itemize}
		\item histogram - min. 256 úrovní jasu pro šedotónové snímky
		\item histogramy pro jednotlivé barevné složky RGB
		\item jasové korekce
		\item jasové transformace - volba a nastavení transformační funkce
		\item ekvalizace histogramu
		\item prahování - ruční prahování zadáním jednoho nebo více prahů
		\item automatické vyhledávání prahu
		\item filtrace obrazu
		\item přebarvování - např. maticí sousednosti
		\item detekce hran
		\item morfologické operace.
	\end{itemize}		
\item{Součástí odevzdané práce bude vypracovaný referát, použité testovací snímky (zadané nebo
vlastní), programová realizace úlohy ve spustitelné verzi, tj. *.EXE, včetně všech potřebných
knihoven a zdrojových souborů programu + prezentace v PowerPointu.}
\item{Semestrální práce bude obsahovat rozbor dosažených výsledků, tzn. zhodnocení jednotlivých
aplikovaných metod, popis jejich pozitivních a negativních vlastností, porovnání výsledků podle
vlastností použitých snímků, srovnání funkce jednotlivých algoritmů, resp. výsledků
a mezivýsledků, s dostupným programovým produktem, např. CVIP Tools,…}
\item{Upřesnění požadavků na semestrální práci podle zadání bude provedeno na přednáškách
a cvičeních ZVI.}
\end{enumerate}
\chapter{Analýza úlohy}
	\section{Předzpracování}
	1. Obrázek 1 (původní). Převod na šedotón kvůli barevnému měřítku, popis snímku, označení objektu podezřelého z konvoluce, bude třeba odstranit/odfiltrovat šum: "vlny" dané materiálem, prach, otisk prstu, škrábance apod.
	2. Histogram - hodně "slepený", většina obrazových bodů nabývá hodnot jasu v intervalu od 100 do 200 (standardní stupnice je od 0 do 255)
	3.	Zvýraznění tmavých odstínů (včetně konvolucí) 
		a) Lineární transformace - posunutí histogramu: "sjednocení" vysokých jasů, eliminace "šumu"; roztažení histogramu: zvýšení kontrastu, došlo ke "zvýraznění" konvolucí, ale i nečistot na vzorku 
		b) Normování histogramu - složité hledání normovací funkce
	\section{Segmentace}
	\section{Detekce konvolucí}


\chapter{Popis implementace}
	Cleaner - vtvoří šedotón, 

\chapter{Uživatelská dokumentace}
	\section{SW požadavky}
	
	\section{Adresářová struktura odevzdávaného souboru}	
	
	\section{Spuštění a ovládání aplikace}	

\chapter{Závěr}

\end{document} 
