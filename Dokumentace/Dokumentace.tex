\documentclass[12pt, a4paper]{report}
\usepackage[utf8]{inputenc}
\usepackage[IL2]{fontenc}
\usepackage[czech]{babel}
\usepackage{graphicx}
\usepackage{epstopdf}
\usepackage{url}

\begin{document}
	\begin{titlepage}
	\includegraphics[width=5cm,natwidth=601,natheight=314]{obrazky/logo.png}
		
	\vspace{4cm}
		\begin {center}
		{\Huge SEMESTRÁLNÍ PRÁCE\\ Z~PŘEDMĚTU KIV/ZVI\\}
		\vspace{1cm}
		{\huge Analýza sekvence mikroskopických snímků, segmentace, detekce objektů\\}
		\end {center}
	\vspace{6cm}
			
	\noindent vypracovali: Denisa Tarantíková, Radek Vais \\
				studijní čísla: A13B0445P, A13B0457P\\
				email:	denitara@students.zcu.cz, vaisr@students.zcu.cz\\
				datum:	23. 6. 2016
	\end{titlepage}

\tableofcontents

\chapter{Úvod}
	Cílem této semestrální práce bude analyzovat sekvenci mikroskopických sním-ků a detekovat v~nich jevy zvané inkluze. Data, která budou použita pro vývoj aplikace, byla snímána infračerveným mikroskopem se zvětšením 5$\times$ (1.09 $\mu$m/px) a krokem mezi řezy (snímky) 0.2 $\mu$m. Snímaným materiálem byla směs kadmia a teluru. 
	
	Inkluze jsou defekty vznikající při tuhnutí taveniny na místech, kde je více atomů teluru/kadmia, než by bylo v~pevné fázi za dané teploty. Z~hlediska energie je výhodnější na těchto místech vytvořit kapalné shluky obohacené o jednu z~komponent. Během tuhnutí povrch defektu zmenšuje svůj objem a přebytečná komponenta vytvoří ve středu defektu koncentrované \uv{kapky}.\footnote{	Zdroj: ŠEDIVÝ, L.: \emph{Difúze přirozených defektů a příměsí v CdTe/CdZnTe.} Praha: Univerzita Karlova v~Praze, Matematicko-fyzikální fakulta, 2012. 84 s. Vedoucí diplomové práce doc. Ing. Eduard Belas, CSc.} Tento jev se může vyskytovat i v~jiných materiálech. 

Výstupem práce bude aplikace pro detekci inkluzí napsaná v~jazyce C++, v jejímž grafickém rozhraní budou označeny detekované objekty.

\chapter{Zadání}
Obecné pokyny a pravidla pro vypracování semestrální práce:
\begin{enumerate}
	\item Součástí semestrální práce je teoretické řešení zadaného úkolu a programová realizace, která může být napsána v libovolném vhodném programovacím jazyku. Programová realizace by měla obsahovat základní funkce:
	\begin{itemize}
		\item{výběr souboru vstupního snímku [formát *.BMP, *.JPG,…] a jeho zobrazení}
		\item{zobrazování dílčích snímků jako výstupů jednotlivých fází zpracování úlohy včetně výsledku konečného}
		\item{volba pořadí jednotlivých metod předzpracování (pokud bude typ zadané úlohy vyžadovat
předzpracování) a vlastního zpracování úlohy dle zadání}
		\item{navržené algoritmy budou optimalizovány podle časového kritéria}
		\item{funkce "Krok zpět", minimálně o 1 krok (podle charakteru úlohy)}
		\item{uložení výsledného snímku do výstupního obrazového souboru, formát viz vstupní snímek}
		\item{uložení výsledných hodnot do výstupních souborů v předepsaném a komentovaném formátu,
např. tabulky hodnot, příznakové vektory,…}
	\end{itemize}
\item{Pokud to zadané téma vyžaduje, bude součástí programové realizace také soubor metod pro
zobrazení globálních charakteristik a předzpracování snímků.}		
\item{Součástí odevzdané práce bude vypracovaný referát, použité testovací snímky (zadané nebo
vlastní), programová realizace úlohy ve spustitelné verzi, tj. *.exe, včetně všech potřebných
knihoven a zdrojových souborů programu + prezentace v PowerPointu.}
\item{Semestrální práce bude obsahovat rozbor dosažených výsledků, tzn. zhodnocení jednotlivých
aplikovaných metod, popis jejich pozitivních a negativních vlastností, porovnání výsledků podle
vlastností použitých snímků, srovnání funkce jednotlivých algoritmů, resp. výsledků
a mezivýsledků, s dostupným programovým produktem, např. CVIP Tools,…}
\end{enumerate}

\chapter{Analýza úlohy}
	\section{Předzpracování}
	Původní snímek (na obrázku 3.1) bude nutné nejdříve převést na šedotónový kvůli barevnému měřítku v~pravé dolní části. Tmavé objekty hvězdicovitého až kulatého tvaru jsou potenciálními inkluzemi (viz obrázek 3.2). Dále lze na snímku pozorovat různé typy šumu: vlny způsobené strukturou materiálu, prach, škrábance, otisky prstů apod. Tyto šumy bude třeba co nejvíce potlačit, aby bylo možné inkluze lépe detekovat.
		
	\begin{figure}[!htb]
	\centering
	\includegraphics[scale=0.4]{obrazky/puvodni.png}
	\label{fig:puvodni}
	\caption{Původní snímek}
	\end{figure}
	
	\begin{figure}[!htb]
	\centering
	\includegraphics[scale=0.12]{obrazky/konvoluce.png}
	\label{fig:inkluze}
	\caption{Inkluze (20násobné zvětšení)}
	\end{figure}	

	Dle histogramu původního snímku na obrázku 3.3 si lze povšimnout, že většina obrazových bodů na stupnici jasu od 0 do 255 nabývá hodnot v~intervalu přibližně od 100 do 200. Budeme uvažovat, že v~ostatních snímcích v~zadané sekvenci je přibližně stejné rozložení hodnot jasu. Vzhledem k~tomu, že inkluze se jeví tmavé v~porovnání s~ostatními obrazovými body snímku, zvýrazněním tmavých odstínů oproti světlým bychom docílili \uv{vystoupení} inkluzí. Pro dosažení tohoto výsledku přebarvíme hodnoty vysokých jasů tak, že jasy vyšší než zvolená mezní hodnota $M$ (včetně) nabudou nejvyššího jasu (255). Tím potlačíme šum, jehož jas nabývá vysokých hodnot (např. vlny). Zbylé hodnoty \uv{roztáhneme} po celém intervalu 0 až 255 vynásobením koeficientem $\frac{256}{M}$, kde v~čitateli je počet hodnot v~celém intervalu a ve jmenovateli počet zbylých (nepřebarvených) hodnot. Touto operací zvýšíme kontrast, a tak usnadníme kontrolu výskytu inkluzí lidským okem. 
	
	\begin{figure}[!htb]
	\centering
	\includegraphics[scale=0.65]{obrazky/puvodni_histogram.png}
	\label{fig:puvodni_histogram}
	\caption{Histogram původního snímku}
	\end{figure}	
	
	Hodnotu, kterou zvolíme jako mezní, určíme na intervalu od 100 do 150 (150 je vrchol histogramu na obrázku 3.3) tak, aby ve snímku zůstaly objekty \uv{podezřelé} z~inkluzí a zároveň bylo potlačeno co nejvíce šumu. Na obrázcích 3.4 až 3.6 jsou výsledné snímky s~mezními hodnotami 100, 150 a 125. Při mezní hodnotě 100 je na snímku stále mnoho viditelných vln, při hodnotě 150 je naopak možné, že některé inkluze \uv{zmizely}. Mezní hodnota 125 se jeví jako dobrý kompromis. Pokud by však byl zvolen jiný materiál, hodnoty jasů by se mohly výrazně lišit. Proto bude vhodné umožnit také ruční nastavení mezní hodnoty.
	
	\begin{figure}[!htb]
	\centering
	\fbox{\includegraphics[scale=0.1]{obrazky/roztazeni_jasu_0_100.png}}
	\label{fig:jas_0_100}
	\caption{Přebarvení vysokých jasů s~mezní hodnotou 100}
	\end{figure}	
	
	\begin{figure}[!htb]
	\centering
	\fbox{\includegraphics[scale=0.1]{obrazky/roztazeni_jasu_0_150.png}}
	\label{fig:jas_0_150}
	\caption{Přebarvení vysokých jasů s~mezní hodnotou 150}
	\end{figure}
	
	\begin{figure}[!htb]
	\centering
	\fbox{\includegraphics[scale=0.1]{obrazky/roztazeni_jasu_0_125.png}}
	\label{fig:jas_0_125}
	\caption{Přebarvení vysokých jasů s~mezní hodnotou 125}
	\end{figure}				
		
	Segmentace snímku bude vzhledem k~množství dat řešena prahováním, které je výpočetně nejméně náročné. Metoda rozplavu je sice výpočetně srovnatelná, ale pro náš případ, kdy potřebujeme zvýraznit krajní hodnoty (0 až mez), \uv{degraduje} na metodu prahování. Hodnota prahu bude nastavena ručně uživatelem z~důvodu možného rozdílného rozložení hodnot jasu v~různých snímcích. V~testovací sekvenci snímků se jako vhodný práh jeví hodnota jasu mezi 160 a 190 (viz obrázky 3.7 a 3.8).
	
	\begin{figure}[!htb]
	\centering
	\fbox{\includegraphics[scale=0.25]{obrazky/prahovani160_vyrez.png}}
	\label{fig:prahovani_160}
	\caption{Prahování s hodnotou prahu 160}
	\end{figure}
	
	\begin{figure}[!htb]
	\centering
	\fbox{\includegraphics[scale=0.25]{obrazky/prahovani190_vyrez.png}}
	\label{fig:prahovani_190}
	\caption{Prahování s hodnotou prahu 190}
	\end{figure}
	
	\section{Detekce inkluzí}
V~jednotlivých předzpracovaných snímcích budou následně detekovány inkluze. Inkluze se na snímcích jeví jako hvězdicovité, téměř kulaté objekty. V~jiných materiálech však mohou mít až trojúhelníkový tvar. Proto bude umožněno nastavení počtu hran inkluzí, které se budou detekovat v~dané sekvenci.

Samotnou detekci je možno provádět několika metodami. Jednou z~nich je vyhledání objektu podle tvarových příznaků, tj. vlastností obrysu a plochy. Výsledek detekce pomocí této metody je na obrázku 3.9. Modře označené oblasti jsou kontury všech nalezených objektů, červeně jsou zvýrazněny objekty, které mají tvarové vlastnosti inkluze.

\begin{figure}[!htb]
	\centering
	\fbox{\includegraphics[scale=0.25]{obrazky/struktura_min3V_pomer1.png}}
	\label{fig:tvarove_priznaky}
	\caption{Inkluze detekované podle tvarových příznaků}
	\end{figure}

Druhou možností je použít kaskádový klasifikátor (běžně používaný pro detekci objektů ve videu, zejména obličejů) naučený detekovat inkluze. Učení klasifikátoru spočívá v~přípravě množiny příkladů s~objekty, které budeme chtít detekovat (tj. pozitivních příkladů), a příkladů pozadí, na kterých již dané objekty nejsou (negativních příkladů). Výsledek detekce inkluzí pomocí kaskádového klasifikátoru je na obrázku 3.10.

\begin{figure}[!htb]
	\centering
	\fbox{\includegraphics[scale=0.25]{obrazky/lepe_nauceny_kalsifikator.png}}
	\label{fig:klasifikator}
	\caption{Inkluze detekované kaskádovým klasifikátorem}
	\end{figure}

\chapter{Popis implementace}
Aplikace je psána v jazyce C++ s~použitím knihovny pro zpracování obrazu OpenCV verze 2.4, která je snadno dostupná na všech platformách, a knihovny Qt pro vytvoření grafického uživatelského rozhraní.
	
	Program je rozdělen do několika modulů, třídy implementující modul grafického uživatelského rozhraní jsou ve složce \texttt{gui} a třídy implementující zpracování snímků jsou ve složce \texttt{core}. V~hlavním modulu (\texttt{main.cpp}) se vytvoří hlavní okno aplikace třídy \texttt{MainWindow}, ve které jsou implementovány veškeré prvky uživatelského rozhraní nacházející se v~hlavním okně. 
	V modulu \texttt{core} jsou implementovány metody pro detekci inkluzí a jiné pomocné třídy. Jako úložiště obrázků zde slouží struktura matic z~OpenCV \texttt{cv::Mat}.

\section{Třída FilenameFactory}
	Tato třída slouží ke snadnému získávání cest k~obrázkům. Inicializačním parametrem třídy je složka (respektive cesta), ve které mají být soubory (obrázky) k~analýze vyhledány. Třída hledá pouze obrázky s~příponu *.tif, protože je to základní formát dat. Nalezené cesty k~souborům jsou seřazeny do fronty a zpracovávány postupem FIFO bez možnosti návratu k~předchozí cestě. Toto řešení bylo zvoleno pro snazší implementaci vícevláknového přístupu. Ovládání běhu analýzy je řízeno pomocí semaforu, který je implementován v~této třídě. Při vstupu do metody \texttt{getRelativePath()} se program pokusí zabrat semafor. V~případě úspěšného získání ho opět uvolní a pokračuje v~provádění kódu. Pokud je ale semafor zabrán, vlákno zde čeká do uvolnění. Tato vedlejší zabrání a uvolnění jsou řízena z~třídy \texttt{MainWindow}.
	
\section{Třída HistogramModifier}
	Tato třída implementuje základní operace nad histogramy obrázků a podporuje posun, roztažení nebo smrštění histogramu snímku. Hodnoty přesahující rozsah jasu (0 - 255) jsou zaokrouhleny na příslušnou hranici.
	
\section{Soubor exception.h}
	Tento hlavičkový soubor a s~ním spojený zdrojový soubor (\texttt{exception.cpp}) obsahují deklarace a implementace výjimek, které jsou způsobeny nestandardními situacemi za běhu programu.  	
	
\section{Třída InclusionDefinition}
	Tato třída představuje popis parametrů hledaných inkluzí pomocí vlastností objektů (například plocha). Součástí implementace této třídy je načítání a generování *.xml souboru s~popisem těchto vlastností. Třída slouží jako přepravka pro parametry vyhledávání a za běhu analýzy existuje v~programu pouze jedna instance. 	
	
\section{Třída StructuralFinder}	
	Tato třída implementuje proces vyhledávání a klasifikaci objektů pomocí jejich vlastností. Třída je potomkem třídy \texttt{QThread}, to umožňuje spustit proces analýzy \uv{na pozadí} nebo rozdělit průběh analýzy mezi více vláken. Vícevláknovou architekturu podporuje i implementace třídy \texttt{FilenameFactory}. Rozvláknění výpočtu nebylo implementováno z~důvodu složité interpretace výsledků velkých dat (tisíce snímků). Jako výstup byl ponechán pouze obraz v~aplikaci.
	Ve třídě je postupně aplikováno několik úprav zdrojového snímku k~dosažení dobrého výsledku. Konkrétně jde o přebarvení všech jasů vyšších než hodnota 150 a následné roztažení histogramu. Na snímek se zlepšeným kontrastem je aplikováno prahování s výchozí hodnotou přebarvovaných jasů 160. Uživatel tuto hodnotu může ovlivnit. Nakonec jsou pomocí hranového detektoru Canny nalezeny obrysy objektů. Posledním krokem je hodnocení, zda nalezené kontury objektů vyhovují definovaným pravidlům.
	Uživatel může definovat všechna výše zmíněná kritéria a jejich kombinace. Objekt bude vyřazen při nesplněni jediného.  
	
\section{Třída HaarFinder}
	Tato třída využívá pro detekci konvolucí naučený HAAR kaskádový klasifikátor implementovaný v knihovně opencv. Opět jde o potomka třídy \texttt{QThread}. Inicializačními parametry jsou odkaz na \texttt{FilenameFactory} a cesta k souboru s naučeným klasifikátorem. Nejlépe naučený klasifikátor je přiložen k programu.
	
\chapter{Uživatelská dokumentace}
	\section{SW požadavky}
	Pro správnou funkčnost aplikace doporučujeme zařízení s~operačním systémem Microsoft Windows verze 7 a vyšší.
	
	\section{Spuštění a ovládání aplikace}	
Po spuštění aplikace dvojklikem na přiložený spustitelný soubor\\ \texttt{InclusionFinder.exe} se objeví hlavní okno (viz obrázek 5.1) s~horním menu, nástrojovou lištou a prostorem, kde lze po načtení složky se sekvencí snímků a spuštění analýzy sledovat jednotlivé snímky, které se právě zpracovávají. Vlevo je zobrazen původní snímek a napravo zpracovaný s~detekovanými inkluzemi. Ve spodní části hlavního okna je posuvník pro nastavení mezní hodnoty prahu, který lze použít v~případě volby metody založené na tvarových příznacích objektů, pro druhou metodu není relevantní.

	\begin{figure}[!htb]
	\centering
	\fbox{\includegraphics[scale=0.32]{obrazky/hlavni_okno.png}}
	\label{fig:hlavni_okno}
	\caption{Hlavní okno aplikace se spuštěnou analýzou}
	\end{figure}

Složka se snímky se načte kliknutím na "Načíst složku..." v~menu "Soubor". Vzhledem k~různému kódování znaků s~diakritikou v~systému Windows může výskyt akcentovaného znaku v~cestě k~souboru způsobit problémy s~programem. Doporučujeme proto umístit načítané soubory do cesty, kde se znaky s~diakritikou nevyskytují. Pro nastavení parametrů hledaných konvolucí slouží okno, které se zobrazí kliknutím na položku "Nastavení" v~menu "Analýza". Samotná analýza se spustí kliknutím na "Spusť analýzu" ve stejném menu nebo na příslušné tlačítko v~liště nástrojů. Pokud byla zvolena metoda kaskádového klasifikátoru, aplikace si před spuštěním analýzy vyžádá cestu ke *.xml souboru s~popisem klasifikátoru, který je v~odevzdávaném souboru přiložen. V~menu "Analýza" se také nachází položky pro načtení či uložení popisu inkluze ze/do souboru ve formátu *.xml. Tento popis se aplikuje na objekty (inkluze), které se budou detekovat. Pod lištou menu se nachází panel nástrojů s~tlačítky pro spuštění či pozastavení analýzy sekvence snímků. Vedle tlačítek je přepínač, jímž lze nastavit metodu pro detekci inkluzí.

	\section{Formát souboru *.xml}
	Součástí aplikace je i možnost nahrát popis vlastností inkluzí ze souboru *.xml. Jeho struktura je následující:\\
	\\
\texttt{<?xml version="1.0" encoding="UTF-8"?>}\\
\texttt{<InclusionDescription>}\\
	\texttt{	<AspectRatio>0.99</AspectRatio>}\\
    \texttt{	<EpsilonRatio>0.19</EpsilonRatio>}\\
    \texttt{	<BoolAspectRatio>1</BoolAspectRatio>}\\
    \texttt{	<MinVertices>2</MinVertices>}\\
    \texttt{	<BoolMinVertices>1</BoolMinVertices>}\\
    \texttt{	<MaxVertices>2147483644</MaxVertices>}\\
    \texttt{	<BoolMaxVertices>1</BoolMaxVertices>}\\
    \texttt{	<Extent>0.99</Extent>}\\
    \texttt{	<EpsilonExtent>1.99</EpsilonExtent>}\\
    \texttt{	<BoolExtent>1</BoolExtent>}\\
    \texttt{	<Note>Poznámka</Note>}\\
\texttt{</InclusionDescription>}\\

	Hodnota \texttt{AspectRatio} znamená poměr stran opsaného obdélníku, který má strany rovnoběžné s~okraji obrázku. \texttt{EpsilonRatio} je přípustná odchylka od požadovaného poměru. \texttt{BoolAspectRatio} je nastaveno na 1, pokud chceme použít popis inkluze pomocí hodnot popsaných výše, v~opačném případě se vepíše 0. Hodnota \texttt{MinVertices} znamená minimální počet vrcholů polygonu aproximujícího hledané inkluze, \texttt{MaxVertices} naopak maximální počet vrcholů. \texttt{BoolMaxVertices} a \texttt{BoolMinVertices} zapínají (1), či vypínají (0) použití příslušných parametrů v detekci. \texttt{Extent} se rovná poměru plochy nalezeného objektu ku ploše obdélníku opsaného tomuto objektu, \texttt{EpsilonExtent} je opět přípustnou odchylkou od této hodnoty a \texttt{BoolExtent} zapíná, nebo vypíná použití těchto parametrů. Do \texttt{Note} lze napsat jakoukoli poznámku k~tomuto nastavení.

\chapter{Závěr}
Cílem této semestrální práce bylo vytvořit aplikaci pro analýzu, segmentaci a detekci objektů v~mikroskopických snímcích. Byla zadána testovací sekvence snímků, ve které jsme hledali jevy zvané inkluze. Výstupem práce je aplikace, která načte danou sekvenci, analyzuje a předzpracuje každý snímek a detekuje inkluze pomocí dvou metod, první z~nich se zakládá na tvarových příznacích, druhou metodou je detekce pomocí naučeného kaskádového klasifikátoru. Nalezené inkluze jsou označeny v~grafickém uživatelském rozhraní aplikace.

Z~hlediska úspěšnosti detekce jsou obě metody v~naší aplikaci srovnatelné, avšak pokud by byl kaskádový klasifikátor naučený na větším množství dat, jeho výsledky by byly mnohem lepší. Kaskádový klasifikátor pracuje v~odstínech šedi, a tak detekuje inkluze již v~\uv{zárodku} (tj. ještě nejsou příliš rozeznatelné), na druhou stranu někdy se vyskytnou detekce falešně pozitivní. Metoda založená na tvarových příznacích je téměř dvakrát rychlejší než klasifikátor a snadno ji lze modifikovat, a tak je možno ji použít pro detekci konvolucí v~jiných materiálech, kde mají jiný tvar.

\end{document} 
