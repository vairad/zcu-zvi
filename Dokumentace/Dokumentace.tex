\documentclass[12pt, a4paper]{report}
\usepackage[utf8]{inputenc}
\usepackage[IL2]{fontenc}
\usepackage[czech]{babel}
\usepackage{graphicx}
\usepackage{epstopdf}
\usepackage{url}

\begin{document}
	\begin{titlepage}
	\includegraphics[width=5cm,natwidth=601,natheight=314]{obrazky/logo.png}
		
	\vspace{4cm}
		\begin {center}
		{\Huge SEMESTRÁLNÍ PRÁCE\\ Z~PŘEDMĚTU KIV/ZVI\\}
		\vspace{1cm}
		{\huge Analýza sekvence mikroskopických snímků, segmentace, detekce objektů\\}
		\end {center}
	\vspace{6cm}
			
	\noindent vypracovali: Denisa Tarantíková, Radek Vais \\
				studijní čísla: A13B0445P, A13B0457P\\
				email:	denitara@students.zcu.cz, vaisr@students.zcu.cz\\
				datum:	14. 6. 2016
	\end{titlepage}

\tableofcontents
\chapter{Úvod}
	Cílem této semestrální práce bude analyzovat sekvenci mikroskopických snímků a detekovat v~nich jevy zvané konvoluce. Data, která budou použita pro vývoj aplikace, byla snímána mikroskopem se zvětšením xy a velikostí řezu abc, snímaným materiálem bylo xyz. Popis konvoluce. Tento jev se vyskytuje i v~jiných materiálech a může nabývat i jiných tvarů. Výstupem práce bude kromě aplikace pro detekci konvolucí napsané v~jazyce C++ i soubor ve formátu *.xml, ve kterém budou uvedeny souřadnice konvolucí ve snímku a názvy snímků, na kterých se vyskytují.
\chapter{Zadání}
Obecné pokyny a pravidla pro vypracování semestrální práce:
\begin{enumerate}
	\item Součástí semestrální práce je teoretické řešení zadaného úkolu a programová realizace, která může být napsána v libovolném vhodném programovacím jazyku. Programová realizace by měla obsahovat základní funkce:
	\begin{itemize}
		\item{výběr souboru vstupního snímku [formát *.BMP, *.JPG,…] a jeho zobrazení}
		\item{zobrazování dílčích snímků jako výstupů jednotlivých fází zpracování úlohy včetně výsledku konečného}
		\item{volba pořadí jednotlivých metod předzpracování (pokud bude typ zadané úlohy vyžadovat
předzpracování) a vlastního zpracování úlohy dle zadání}
		\item{navržené algoritmy budou optimalizovány podle časového kritéria}
		\item{funkce "Krok zpět", minimálně o 1 krok (podle charakteru úlohy)}
		\item{uložení výsledného snímku do výstupního obrazového souboru, formát viz vstupní snímek}
		\item{uložení výsledných hodnot do výstupních souborů v předepsaném a komentovaném formátu,
např. tabulky hodnot, příznakové vektory,…}
	\end{itemize}
\item{Pokud to zadané téma vyžaduje, bude součástí programové realizace také soubor metod pro
zobrazení globálních charakteristik a předzpracování snímků.}		
\item{Součástí odevzdané práce bude vypracovaný referát, použité testovací snímky (zadané nebo
vlastní), programová realizace úlohy ve spustitelné verzi, tj. *.exe, včetně všech potřebných
knihoven a zdrojových souborů programu + prezentace v PowerPointu.}
\item{Semestrální práce bude obsahovat rozbor dosažených výsledků, tzn. zhodnocení jednotlivých
aplikovaných metod, popis jejich pozitivních a negativních vlastností, porovnání výsledků podle
vlastností použitých snímků, srovnání funkce jednotlivých algoritmů, resp. výsledků
a mezivýsledků, s dostupným programovým produktem, např. CVIP Tools,…}
\end{enumerate}
\chapter{Analýza úlohy}
	\section{Předzpracování}
	1. Obrázek 1 (původní). Převod na šedotón kvůli barevnému měřítku, popis snímku, označení objektu podezřelého z konvoluce, bude třeba odstranit/odfiltrovat šum: "vlny" dané materiálem, prach, otisk prstu, škrábance apod.
	2. Histogram - hodně "slepený", většina obrazových bodů nabývá hodnot jasu v intervalu od 150 do 200 (standardní stupnice je od 0 do 255)
	3.	Zvýraznění tmavých odstínů (včetně konvolucí) 
		a) Lineární transformace - posunutí histogramu: "sjednocení" vysokých jasů, eliminace "šumu"; roztažení histogramu: zvýšení kontrastu, došlo ke "zvýraznění" konvolucí, ale i nečistot na vzorku 
		b) Normování histogramu - složité hledání normovací funkce
	\section{Segmentace}
	\section{Detekce konvolucí}


\chapter{Popis implementace}
Volba programových prostředků: OpenCV verze 2.4 (hlavní verze, snadno dostupná na všech platformách, doporučení???)
	Cleaner - vtvoří šedotón, 

\chapter{Uživatelská dokumentace}
	\section{SW požadavky}
	
	\section{Adresářová struktura odevzdávaného souboru}	
	
	\section{Spuštění a ovládání aplikace}	

\chapter{Závěr}
Cílem této semestrální práce bylo vytvořit aplikaci pro detekci konvolucí.

\end{document} 
