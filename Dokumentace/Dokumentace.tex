\documentclass[12pt, a4paper]{report}
\usepackage[utf8]{inputenc}
\usepackage[IL2]{fontenc}
\usepackage[czech]{babel}
\usepackage{graphicx}
\usepackage{epstopdf}
\usepackage{url}

\begin{document}
	\begin{titlepage}
	\includegraphics[width=5cm,natwidth=601,natheight=314]{logo.png}
		
	\vspace{4cm}
		\begin {center}
		{\Huge SEMESTRÁLNÍ PRÁCE\\ Z PŘEDMĚTU KIV/ZVI\\}
		{\huge Analýza sekvence mikroskopických snímků, segmentace, detekce objektů\\}
		\end {center}
	\vspace{7cm}
			
	\noindent vypracovali: Denisa Tarantíková, Radek Vais \\
				studijní čísla: A13B0445P, A13B0457P\\
				email:	denitara@students.zcu.cz, vaisr@students.zcu.cz\\
				datum:	10. 6. 2016
	\end{titlepage}

\tableofcontents
\chapter{Úvod}
	
\chapter{Zadání}
Obecné pokyny a pravidla pro vypracování semestrální práce
1. Semestrální práce bude vypracována na téma podle zadání.
2. Součástí semestrální práce je teoretické řešení zadaného úkolu a programová realizace, která
může být napsána v libovolném vhodném programovacím jazyku (Matlab nedoporučujeme).
3. Programová realizace by měla obsahovat základní funkce:
- výběr souboru vstupního snímku [ formát *.BMP, *.JPG, … ] a jeho zobrazení,
- zobrazování dílčích snímků jako výstupů jednotlivých fází zpracování úlohy včetně výsledku
konečného,
- volba pořadí jednotlivých metod předzpracování ( pokud bude typ zadané úlohy vyžadovat
předzpracování ) a vlastního zpracování úlohy dle zadání,
- navržené algoritmy budou optimalizovány podle časového kritéria,
- funkce "Krok zpět", minimálně o 1 krok ( podle charakteru úlohy ),
- uložení výsledného snímku do výstupního obrazového souboru, formát viz vstupní snímek,
- uložení výsledných hodnot do výstupních souborů v předepsaném a komentovaném formátu,
např. tabulky hodnot, příznakové vektory, …
4. Pokud to zadané téma vyžaduje, bude součástí programové realizace také soubor metod pro
zobrazení globálních charakteristik a předzpracování snímků, tj. metody typu:
- histogram - min. 256 úrovní jasu pro šedotónové snímky
- histogramy pro jednotlivé barevné složky RGB
- jasové korekce
- jasové transformace - volba a nastavení transformační funkce
- ekvalizace histogramu
- prahování - ruční prahování zadáním jednoho nebo více prahů
- automatické vyhledávání prahu
- filtrace obrazu
- přebarvování - např. maticí sousednosti
- detekce hran
- morfologické operace.
5. Součástí odevzdané práce bude vypracovaný referát, použité testovací snímky ( zadané nebo
vlastní ), programová realizace úlohy ve spustitelné verzi, tj. *.EXE, včetně všech potřebných
knihoven a zdrojových souborů programu + prezentace v PowerPointu.
6. Semestrální práce bude obsahovat rozbor dosažených výsledků, tzn. zhodnocení jednotlivých
aplikovaných metod, popis jejich pozitivních a negativních vlastností, porovnání výsledků podle
vlastností použitých snímků, srovnání funkce jednotlivých algoritmů, resp. výsledků
a mezivýsledků, s dostupným programovým produktem, např. CVIP Tools, ….
7. Upřesnění požadavků na semestrální práci podle zadání bude provedeno na přednáškách
a cvičeních ZVI.

\chapter{Analýza úlohy}
	\section{Předzpracování}
	\section{Segmentace}
	\section{Detekce konvolucí}


\chapter{Popis implementace}

\chapter{Uživatelská dokumentace}
	\section{SW požadavky}
	
	\section{Adresářová struktura odevzdávaného souboru}	
	
	\section{Spuštění a ovládání aplikace}	

\chapter{Závěr}

\end{document} 
