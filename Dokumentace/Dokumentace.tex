\documentclass[12pt, a4paper]{report}
\usepackage[utf8]{inputenc}
\usepackage[IL2]{fontenc}
\usepackage[czech]{babel}
\usepackage{graphicx}
\usepackage{epstopdf}
\usepackage{url}

\begin{document}
	\begin{titlepage}
	\includegraphics[width=5cm,natwidth=601,natheight=314]{obrazky/logo.png}
		
	\vspace{4cm}
		\begin {center}
		{\Huge SEMESTRÁLNÍ PRÁCE\\ Z~PŘEDMĚTU KIV/ZVI\\}
		\vspace{1cm}
		{\huge Analýza sekvence mikroskopických snímků, segmentace, detekce objektů\\}
		\end {center}
	\vspace{6cm}
			
	\noindent vypracovali: Denisa Tarantíková, Radek Vais \\
				studijní čísla: A13B0445P, A13B0457P\\
				email:	denitara@students.zcu.cz, vaisr@students.zcu.cz\\
				datum:	15. 6. 2016
	\end{titlepage}

\tableofcontents
\chapter{Úvod}
	Cílem této semestrální práce bude analyzovat sekvenci mikroskopických snímků a detekovat v~nich jevy zvané konvoluce. Data, která budou použita pro vývoj aplikace, byla snímána mikroskopem se zvětšením xy a velikostí řezu abc, snímaným materiálem bylo xyz. Popis konvoluce. Tento jev se vyskytuje i v~jiných materiálech a může nabývat i jiných tvarů. Výstupem práce bude kromě aplikace pro detekci konvolucí napsané v~jazyce C++ i soubor ve formátu *.xml, ve kterém budou uvedeny souřadnice konvolucí ve snímku a názvy snímků, na kterých se vyskytují.
\chapter{Zadání}
Obecné pokyny a pravidla pro vypracování semestrální práce:
\begin{enumerate}
	\item Součástí semestrální práce je teoretické řešení zadaného úkolu a programová realizace, která může být napsána v libovolném vhodném programovacím jazyku. Programová realizace by měla obsahovat základní funkce:
	\begin{itemize}
		\item{výběr souboru vstupního snímku [formát *.BMP, *.JPG,…] a jeho zobrazení}
		\item{zobrazování dílčích snímků jako výstupů jednotlivých fází zpracování úlohy včetně výsledku konečného}
		\item{volba pořadí jednotlivých metod předzpracování (pokud bude typ zadané úlohy vyžadovat
předzpracování) a vlastního zpracování úlohy dle zadání}
		\item{navržené algoritmy budou optimalizovány podle časového kritéria}
		\item{funkce "Krok zpět", minimálně o 1 krok (podle charakteru úlohy)}
		\item{uložení výsledného snímku do výstupního obrazového souboru, formát viz vstupní snímek}
		\item{uložení výsledných hodnot do výstupních souborů v předepsaném a komentovaném formátu,
např. tabulky hodnot, příznakové vektory,…}
	\end{itemize}
\item{Pokud to zadané téma vyžaduje, bude součástí programové realizace také soubor metod pro
zobrazení globálních charakteristik a předzpracování snímků.}		
\item{Součástí odevzdané práce bude vypracovaný referát, použité testovací snímky (zadané nebo
vlastní), programová realizace úlohy ve spustitelné verzi, tj. *.exe, včetně všech potřebných
knihoven a zdrojových souborů programu + prezentace v PowerPointu.}
\item{Semestrální práce bude obsahovat rozbor dosažených výsledků, tzn. zhodnocení jednotlivých
aplikovaných metod, popis jejich pozitivních a negativních vlastností, porovnání výsledků podle
vlastností použitých snímků, srovnání funkce jednotlivých algoritmů, resp. výsledků
a mezivýsledků, s dostupným programovým produktem, např. CVIP Tools,…}
\end{enumerate}
\chapter{Analýza úlohy}
	\section{Předzpracování}
	Původní snímek (na obrázku 3.1) bude nutné nejdříve převést na šedotónový kvůli barevnému měřítku v~pravé dolní části. Tmavé objekty hvězdicovitého až kulatého tvaru jsou potenciálními konvolucemi (na obrázku 3.2 lze vidět snímek konvoluce ve 20násobném zvětšení). Dále lze na snímku pozorovat různé typy šumu: vlny způsobené strukturou materiálu, prach, škrábance, otisky prstů apod. Tyto šumy bude třeba co nejvíce potlačit, aby bylo možné konvoluce lépe detekovat.
		
	\begin{figure}[!htb]
	\centering
	\includegraphics[scale=0.4]{obrazky/puvodni.png}
	\label{fig:puvodni}
	\caption{Původní snímek}
	\end{figure}
	
	\begin{figure}[!htb]
	\centering
	\includegraphics[scale=0.12]{obrazky/konvoluce.png}
	\label{fig:konvoluce}
	\caption{Konvoluce}
	\end{figure}	

	Dle histogramu původního snímku na obrázku 3.3 si lze povšimnout, že většina obrazových bodů na stupnici jasu od 0 do 255 nabývá hodnot v~intervalu přibližně od 100 do 200. Budeme uvažovat, že v~ostatních snímcích v~zadané sekvenci je přibližně stejné rozložení hodnot jasu. Vzhledem k~tomu, že konvoluce se jeví tmavé v~porovnání s~ostatními obrazovými body snímku, zvýrazněním tmavých odstínů oproti světlým bychom docílili \uv{vystoupení} konvolucí. Pro dosažení tohoto výsledku přebarvíme hodnoty vysokých jasů tak, že jasy vyšší než zvolená mezní hodnota $M$ (včetně) nabudou nejvyššího jasu (255). Tím potlačíme šum, jehož jas nabývá vysokých hodnot (např. vlny). Zbylé hodnoty \uv{roztáhneme} po celém intervalu 0 až 255 vynásobením koeficientem $\frac{256}{M}$, kde v~čitateli je počet hodnot v~celém intervalu a ve jmenovateli počet zbylých (\uv{nesjednocovaných}) hodnot. Touto operací zvýšíme kontrast, a tak usnadníme kontrolu výskytu konvolucí lidským okem. Hodnotu, kterou zvolíme jako mezní, určíme na intervalu od 100 do 150 (150 je vrchol histogramu na obrázku 3.3) tak, aby ve snímku zůstaly objekty \uv{podezřelé} z~konvolucí a zároveň bylo potlačeno co nejvíce šumu. Na obrázcích 3.4 až 3.6 vidíme výsledné snímky s~mezními hodnotami 100, 150 a 125. Při mezní hodnotě 100 je na snímku stále mnoho viditelných vln, při hodnotě 150 je naopak možné, že některé konvoluce \uv{zmizely}. Mezní hodnota 125 se jeví jako dobrý kompromis. Pokud by však byl zvolen jiný materiál, hodnoty jasů by se mohly výrazně lišit. Proto bude vhodné umožnit také ruční nastavení mezní hodnoty.
	
	\begin{figure}[!htb]
	\centering
	\includegraphics[scale=0.65]{obrazky/puvodni_histogram.png}
	\label{fig:puvodni_histogram}
	\caption{Histogram původního snímku}
	\end{figure}
	
	\begin{figure}[!htb]
	\centering
	\fbox{\includegraphics[scale=0.1]{obrazky/roztazeni_jasu_0_100.png}}
	\label{fig:jas_0_100}
	\caption{Přebarvení vysokých jasů s~mezní hodnotou 100 a roztažení}
	\end{figure}	
	
	\begin{figure}[!htb]
	\centering
	\fbox{\includegraphics[scale=0.1]{obrazky/roztazeni_jasu_0_150.png}}
	\label{fig:jas_0_150}
	\caption{Přebarvení vysokých jasů s~mezní hodnotou 150 a roztažení}
	\end{figure}
	
	\begin{figure}[!htb]
	\centering
	\fbox{\includegraphics[scale=0.1]{obrazky/roztazeni_jasu_0_125.png}}
	\label{fig:jas_0_125}
	\caption{Přebarvení vysokých jasů s~mezní hodnotou 125 a roztažení}
	\end{figure}				
		
	Segmentace snímku bude vzhledem k~množství dat řešena prahováním, které je výpočetně nejméně náročné. Metoda rozplavu je sice výpočetně srovnatelná, ale pro náš případ, kdy potřebujeme zvýraznit krajní hodnoty (0 až mez), \uv{degraduje} na metodu prahování. Hodnotu prahu nastaví uživatel z~důvodu možného rozdílného rozložení hodnot jasu. V~testovací sekvenci snímků se jako vhodný práh jeví hodnota jasu mezi 160 a 190.
	\section{Detekce konvolucí}


\chapter{Popis implementace}
Volba programových prostředků: OpenCV verze 2.4 (hlavní verze, snadno dostupná na všech platformách, doporučení???)
	Cleaner - vtvoří šedotón, 

\chapter{Uživatelská dokumentace}
	\section{SW požadavky}
	
	\section{Adresářová struktura odevzdávaného souboru}	
	
	\section{Spuštění a ovládání aplikace}	

\chapter{Závěr}
Cílem této semestrální práce bylo vytvořit aplikaci pro detekci konvolucí.

\end{document} 
